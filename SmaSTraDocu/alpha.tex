Das Ziel der Alpha-Version war es eine Grundlage für die SmaSTra-Anwendung zu schaffen. Diese Grundlage beinhaltete drei Hauptkomponenten:

\begin{itemize}
\item Eine Kern-Bibliothek in Android-Java, die das Ausführen von Generierten Transformationen auf Android Geräten ermöglicht.
\item Eine grafische Benutzeroberfläche, die das erstellen von Transformationsbäumen ermöglicht.
\item Eine Bibliothek für die Generierung von Android-Java Code aus den erstellten Transformationsbäumen.
\end{itemize}

Die Kern-Bibliothek beinhaltete Basisklassen für die Verwendung im generierten Code und grundlegende Transformationsmethoden, die beim erstellen von Transformationsbäumen genutzt werden konnten.

Die Benutzeroberfläche bot in der Alpha-Version bereits die Möglichkeit einen Transformationsbaum zu erstellen und mit Hilfe der Generierungsbibliothek in Android Code zu übersetzen. Komfortfunktionen wie Zoomen und das automatische Sortieren von Bäumen war zu diesem Zeitpunkt geplant allerdings noch nicht umgesetzt. Auch das Erstellen von eigenen Transformationsmethoden innerhalb der grafischen Benutzeroberfläche, das Abspeichern oder Teilen von Transformationsbäumen wurde in der Alphaversion noch nicht unterstützt.